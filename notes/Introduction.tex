\chapter{Introduction}
\label{sec:org7d7a9e6}
\section{The Idea}
\label{sec:org0041e00}

The Digital Life Tracking App is designed to enable users to track their daily tasks and the time spent on them.
To make it more convenient to trigger an activity, the start and end of an activity may be triggered in different ways. 
Possible activation mechanisms could be:
\begin{itemize}
\item Two claps
\item moving the smartphone in a certain way (gestures)
\item activation based on GPS position
\item pressing a button
\end{itemize}

Users should be able to define their own activities and select an activation function from a list and assign it to one of their activities.
The user should then be able to display his history in form of charts.
A separate device (particle photon board) could also be used to trigger a specific activity.

\subsection{The Deep Dive}
\label{sec:orgf0412a9}
For the prototype in the context of the assigment we decided to focus mainly on the detection of double clapping as an activity trigger. From the original idea a concept for the following app was developed:

\begin{center}
	\textbf{{\Large Digital Life Tracking}}
\end{center}
\begin{figure}[H]
	\centering
	\includegraphics[width=0.5\linewidth]{./imgs/appIcon.png}
\end{figure}
The app should allow a user to divide his daily routine into categories of activities and to measure the time required for each activity. This enables him to get an overview of his time invested in various activities.