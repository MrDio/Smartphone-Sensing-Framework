\chapter*{Abstract}
The aim of this work is to generate an application that starts a timer that measures the time of a certain activity by clapping into the hands. The timer is stopped by the repeated clapping and the time is saved together with a corresponding activity. This creates a history for the user in which he can see how long which activity was performed.
\\
The problem to be investigated is the proper detection of clapping noises. It must be possible to distinguish between background noises such as coughing or knocking. This presents a major challenge, as they are difficult to distinguish from a technical point of view.
\\
A large number of sound samples were recorded for clapping, tapping, coughing and speaking and the respective frequency spectrum was compared. In order to get a variation of the sounds different people have executed these tasks. In addition, various software tools are used to analyse the frequencies.
\\
The result of this work is that the analysed noises partly happen on the same frequencies. %Depending on who's making the noise. 
A distinction can therefore only be made in the duration of the tones. Using individual algorithms for this would become too complex, as many cases have to be covered. It therefore makes more sense to use pattern recognition frameworks. %However, they need a lot of test dates.
\\
When enough time is available, the use of pattern recognition frameworks makes the most sense. The devices that were used reliably detect the noises and the mathematical calculations are also possible without problems on the systems.