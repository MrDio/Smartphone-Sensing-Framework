\begin{abstract}
LapOps is and Android App about recognising race tracks with its sections and improving racing performance by giving tips based on the performance analysation of each section.

Therefore two sensors of a smartphone had to be used. By measuring forces on the car with an accelerometer and identify different laps with the activation of an proximeter triggered by a bridge at the start of each lap, a race track can be identified surprisingly precise.

Since the identification of sections was not possible to realise in real-time given the early deadline of the project, the scope had to be set to storing the race data and analyse the data after finishing the race.

This is done by capturing the forces acting on the car, remove the noise as good as possible and differentiate patterns, what does raise problems since not every round is driven validly or shows strange occurrences in the data set. These problems are equalized by testing different threshold parameters and using the best analysation outcome.

After the identification of sections is completed, these sections are classified into different categories. This way each lap is divided into sections matching the data collected.

The laps are then compared to each other and tips regarding the performance improvement in each section is given. These tips are gathered inside a report. Which can be reviewed at any time. 
\end{abstract}